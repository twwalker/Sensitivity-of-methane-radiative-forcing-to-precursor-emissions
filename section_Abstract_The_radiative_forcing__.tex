\section*{Abstract}

The radiative forcing from methane responds both to direct emissions of methane and to changes in its chemical loss rate due to sustained changes in the atmospheric abundance of short-lived precursor gases NOx, CO, and non-methane hydrocarbons (NMHCs). We use the adjoint of the GEOS-Chem chemical transport model to calculate the spatially-resolved sensitivity of methane radiative forcing to changes in \color{anthropogenic sources of precursor emissions}. The normalized annual global mean sensitivity of the methane lifetime to land-based anthropogenic emissions of NOx, CO, and NMHC emissions is -0.145, 0.047, and 0.027 respectively. We apply the spatially-resolved sensitivities to a spatially explicit projection of methane and precursor emissions from RCP 6.0 to calculate the total change in methane radiative forcing attributable to local changes in emissions. The top 10\% of locations with positive radiative forcing responses to all emissions changes at 2050 combined cause 50\% of the global positive radiative forcing, and the top 10\% of locations with negative radiative forcing responses cause 60\% of the global negative radiative forcing. We examine the radiative forcing response in China in particular, considering how deviations from the RCP emissions trajectories will impact the response given the new knowledge of the spatially-resolved sensitivities.