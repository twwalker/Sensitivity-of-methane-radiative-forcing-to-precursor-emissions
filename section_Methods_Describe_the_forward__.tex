\section{Methods}

Describe the forward model as well: chemistry, dynamics, resolution...

The GEOS-Chem adjoint model~\cite{ref:henze2007} (v35) is based on the forward GEOS-Chem model v9-02 (\url{http://www.geos-chem.org}). It permits efficient calculation of the gradient of a metric of model outputs, such as concentrations, to all model inputs simultaneously, including emissions and reaction rates~\cite{ref:walker2015}. Let us define the global instantaneous methane loss rate $L$ as

\begin{equation}
L=\sum_{i \in D} \mathrm{k}_i \mathrm{[OH]}_i \mathrm{[CH_4]}_i,
\end{equation}
where the subscript $i$ denotes each model gridbox within the tropospheric domain $D$, and $\mathrm{k}_i$ is the temperature-dependent bimolecular rate constant for hydroxyl-driven methane loss~\citep{ref:sander2011}. The sensitivity of this quantity to precursor emissions $\mathrm{E}$, which is calculated by the adjoint model, is

\begin{equation}
\lambda = \nabla_\mathrm{E} L = \sum_{i \in D} \mathrm{k}_i \mathrm{[CH_4]}_i \frac{\partial \mathrm{[OH]_i}}{\partial \mathrm{E}}.
\end{equation}

A change in the loss rate of methane affects the global mean methane concentration $C$, after accounting for the feedback effect that methane concentrations have on its own lifetime~\citep{ref:naik2005}:

\begin{equation}
\Delta C = f\big(1 - \frac{L}{L+\nabla_E L \delta E}\big).
\end{equation}

This change can then be related to a change in the methane radiative forcing $F$~\citep{ref:myhre1998}:

\begin{equation}
\Delta F = \alpha (\sqrt{C-\Delta C} - \sqrt{C})
\end{equation}

These relations combine to give the sensitivity of the methane radiative forcing to a change in precursor emissions:

\begin{equation}
\frac{\Delta F}{\Delta E} = \frac{\Delta F}{\Delta C} \frac{\Delta C}{\Delta L} \frac{\Delta L}{\Delta E} = .
\end{equation}

The adjoint-derived sensitivities are combined with emissions changes drawn from representative concentration pathways (RCP's,~\citet{ref:vanvuuren2011}). We assume that the sensitivity calculation is independent of the underlying emissions (i.e. that the spatial pattern of the sensitivity of methane to emissions does not change significantly if the precursor emissions change). These emissions serve as an example, as the adjoint sensitivities could be applied to any (reasonable?) change in emissions.