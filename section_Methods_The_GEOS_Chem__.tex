\section{Methods}

The GEOS-Chem global chemical transport model (\url{http://www.geos-chem.org/}) describes the evolution of atmospheric chemical composition, under transport driven by assimilated meteorological fields from the NASA Goddard Earth Observing System (GEOS-5). The resolution of the GEOS-5 fields is degraded to $2^{\circ} \times 2.5^{\circ}$ and 47 vertical layers for computational efficiency. The chemical mechanism in GEOS-Chem simulates the NOx-HOx-NMHC chemistry of the troposphere~\citep{ref:mao2010}. Methane concentrations are prescribed in four zonal bands with values derived from flask data for the simulation year.

The GEOS-Chem adjoint model~\cite{ref:henze2007} (v35h) is based on the forward GEOS-Chem model and incorporates forward model updates up to version v9-02-01. It permits efficient calculation of the gradient of a metric of model outputs, such as concentrations, to all model inputs simultaneously, including emissions and reaction rates~\cite{ref:paulot2012}. We use the adjoint gradients as a means of attributing changes in composition to grid-scale emissions, as described below. Gradients, interpreted as sensitivities, from the GEOS-Chem adjoint model have previously been used in attribution studies of tropospheric composition~\citep{ref:zhang2009,ref:walker2012} and direct ozone radiative forcing~\citep{ref:bowman2012}.

Let us define the global instantaneous methane loss rate to tropospheric OH, $L$, as

\begin{equation}
L=\sum_{i \in D} \mathrm{k}_i \mathrm{[OH]}_i \mathrm{[CH_4]}_i,
\label{eqn:lch4}
\end{equation}

where the subscript $i$ denotes each model grid box within the tropospheric domain $D$, and $\mathrm{k}_i$ is the temperature-dependent bimolecular rate constant for hydroxyl-driven methane loss~\citep{ref:sander2011}. The sensitivity $\lambda$ of the loss rate defined in Equation~\ref{eqn:lch4} to grid scale precursor emissions $\mathbf{E}$, which is calculated by the GEOS-Chem adjoint model, is

\begin{equation}
\lambda = \nabla_\mathbf{E} L = \sum_{i \in D} \mathrm{k}_i \mathrm{[CH_4]}_i \frac{\partial \mathrm{[OH]_i}}{\partial \mathbf{E}}.
\label{eqn:dlde}
\end{equation}

A change in the loss rate of methane affects the global mean methane concentration $C$. After accounting for the feedback effect of methane on its own lifetime~\citep{ref:fuglestvedt1999,ref:naik2005}, the concentration and loss are related as follows

\begin{equation}
\begin{align}
\begin{split}
\Delta C &= C^{\prime} - C \\
         &= f C \Big(1 - \frac{L}{L^{\prime}}\Big) \\
         &= fC\Big(\frac{\Delta L}{L+\Delta L}\Big),
\end{split}
\label{eqn:dc}
\end{align}
\end{equation}

where the primed variables denote those calculated in a chemical state perturbed by additional precursor emissions (e.g. $\Delta L = L^{\prime}-L$).  We adopt the value $f=1.34$ for the feedback factor of additional methane on its own lifetime~\citep{ref:holmes2013}. The sensitivity of the change in concentration to a change in the loss rate can be found by taking the partial derivative of Equation~\ref{eqn:dc}

\begin{equation}
\frac{\partial (\Delta C)}{\partial (\Delta L)} = \frac{fCL}{(L+\Delta L)^2}.
\label{eqn:dcdl}
\end{equation}

This change can then be related to a change in the methane radiative forcing $F$ using the simplified expression from~\citet{ref:myhre1998}:

\begin{equation}
\Delta F = \alpha (\sqrt{C-\Delta C} - \sqrt{C}),
\label{eqn:df}
\end{equation}

where $\alpha=0.036$ $\mathrm{W}\mathrm{m}^{-2}\mathrm{ppbv}^{-1/2}$~\citep{ref:myhre1998} and changes in atmospheric N2O abundance have been neglected. Differentiating Equation~\ref{eqn:df} gives the sensitivity of the forcing to a change in methane concentration,

\begin{equation}
\frac{\partial (\Delta F)}{\partial (\Delta C)} = \frac{-\alpha}{2\sqrt{C-\Delta C}}.
\label{eqn:dfdc}
\end{equation}

Combinining Equations~\ref{eqn:dlde},~\ref{eqn:dcdl}, and~\ref{eqn:dfdc} relates the sensitivity of the methane radiative forcing to a change in precursor emissions, through their interaction with OH and a change in methane loss rate:

\begin{equation}
\begin{align}
\begin{split}
\frac{\Delta F}{\Delta E} &= \frac{\Delta F}{\Delta C} \frac{\Delta C}{\Delta L} \nabla_{\mathbf{E}} L \\
                          &= \frac{-\lambda \alpha f \sqrt{C}}{2L}.
\end{split}
\label{eqn:dfde}
\end{align}
\end{equation}

Here we have assumed that the changes in methane loss rate and concentration are small compared to their magnitude (e.g. $\Delta C \ll C$ and $\Delta L \ll L$).

To derive the RF due to methane emissions, we assume the methane emissions become globally well-mixed in the atmosphere, and calculate the change in concentration produced for a given mass of additional emissions. Then we apply Equation~\ref{eqn:df} to calculate the forcing. We invert this relationship to calculate the equivalent emissions of methane due to a change in RF from changes in precursor emissions.

The adjoint-derived sensitivities are combined with emissions changes drawn from representative concentration pathways (RCP's,~\citet{ref:lamarque2011,ref:vanvuuren2011}) to estimate projected changes in methane RF due to precursor emission changes. We assume that the sensitivity calculation is independent of the underlying emissions (i.e. that the spatial pattern of the sensitivity of methane to emissions does not change significantly if the precursor emissions change), and therefore that the sensitivities calculated for the present day atmosphere may be applied to a projected emissions scenario. RF due to direct emissions of methane are calculated assuming the methane becomes well-mixed through an application of Equation~\ref{eqn:df}. The RCP emissions serve as an example, as the above sensitivity method could be applied to any reasonable change in emissions.

Figure~\ref{fig:rfsens} shows the sensitivity of global methane RF in July to emissions of various precursor gases. These sensitivities describe the change in methane RF to a fractional change in precursor emissions at a particular location. Sensitivities to anthropogenic NOx are generally larger at low latitudes in the summer hemisphere, where hydroxyl radical concentrations are highest. Certain locations provide more favorable access for short-lived NOx to reach the global atmosphere, through either high altitude (e.g. Mexico City or Bogot\'a) or strong convection (e.g. southeastern Asia), giving emissions in those locations a stronger influence on the global mean methane loss rate~\citep{ref:bowman2012}. The highest sensitivities do not in general correspond to the locations of largest emissions for NOx, but the tendency for larger values in the tropics could have implications as population growth and economic development move equatorward. The annual mean sensitivity for each precursor is used in the analysis presented in this work.

