\section{Methods}

The GEOS-Chem global chemical transport model (\url{http://www.geos-chem.org}) describes the evolution of atmospheric chemical composition, under transport driven by assimilated meteorological fields from the NASA Goddard Earth Observing System (GEOS-5). The resolution of the GEOS-5 fields is degraded to $2^{\circ} \times 2.5^{\circ}$ and 47 vertical layers for computational efficiency. The chemical mechanism in GEOS-Chem simulates NOx-HOx-VOC chemistry of the troposphere. Methane concentrations are prescribed in four zonal bands with values taken from flask data for the simulation year.

% compare ohsens.2 to ohsens.3 to find difference in loss rate between zonal methane and Turner average methane

The GEOS-Chem adjoint model~\cite{ref:henze2007} (v35h) is based on the forward GEOS-Chem model v9-02. It permits efficient calculation of the gradient of a metric of model outputs, such as concentrations, to all model inputs simultaneously, including emissions and reaction rates~\cite{ref:walker2016}. The GEOS-Chem adjoint model has previously been used for studies of tropospheric composition~\citep{ref:zhang2009, ref:walker2012} and radiative forcing~\citep{ref:bowman2012}.

Let us define the global instantaneous methane loss rate to tropospheric OH, $L$, as

\begin{equation}
L=\sum_{i \in D} \mathrm{k}_i \mathrm{[OH]}_i \mathrm{[CH_4]}_i,
\end{equation}

where the subscript $i$ denotes each model grid box within the tropospheric domain $D$, and $\mathrm{k}_i$ is the temperature-dependent bimolecular rate constant for hydroxyl-driven methane loss~\citep{ref:sander2011}. The sensitivity $\lambda$ of this quantity to grid scale precursor emissions $\mathbf{E}$, which is calculated by the adjoint model, is

\begin{equation}
\lambda = \nabla_\mathbf{E} L = \sum_{i \in D} \mathrm{k}_i \mathrm{[CH_4]}_i \frac{\partial \mathrm{[OH]_i}}{\partial \mathbf{E}}.
\end{equation}

A change in the loss rate of methane affects the global mean methane concentration $C$. Let us denote with a prime variables calculated in this perturbed state. After accounting for the feedback effect of methane on its own lifetime~\citep{ref:fuglestvedt1999,ref:naik2005}, the concentration and loss are related as follows

\begin{equation}
\Delta C = C^{\prime} - C = f C \Big(1 - \frac{L}{L^{\prime}}\Big)
\end{equation}

\begin{equation}
\Delta C = fC\Big(\frac{\Delta L}{L+\Delta L}\Big),
\end{equation}

where $\Delta L = L^{\prime}-L$, and we adopt the value $f=1.34$ for the feedback factor~\citep{ref:holmes2013}. The sensitivity of the change in concentration to a change in the loss rate can be found by taking the partial derivative

\begin{equation}
\frac{\partial (\Delta C)}{\partial (\Delta L)} = \frac{fCL}{(L+\Delta L)^2}.
\end{equation}

This change can then be related to a change in the methane radiative forcing $F$ using the simplified expression from~\citet{ref:myhre1998}:

\begin{equation}
\Delta F = \alpha (\sqrt{C-\Delta C} - \sqrt{C}),
\end{equation}

where $\alpha=0.036$ and changes in N2O have been neglected. Differentiating to get the sensitivity to a change in methane concentration,

\begin{equation}
\frac{\partial (\Delta F)}{\partial (\Delta C)} = \frac{-\alpha}{2\sqrt{C-\Delta C}}.
\end{equation}

These relations combine to give the sensitivity of the methane radiative forcing to a change in precursor emissions:

\begin{equation}
\frac{\Delta F}{\Delta E} = \frac{\Delta F}{\Delta C} \frac{\Delta C}{\Delta L} \nabla_{\mathbf{E}} L = \frac{-\lambda \alpha f \sqrt{C}}{2L}.
\end{equation}

Here we have assumed that the changes in loss rate and concentration are small compared to their magnitude (e.g. $\Delta C \ll C$ and $\Delta L \ll L$).

The adjoint-derived sensitivities are combined with emissions changes drawn from representative concentration pathways (RCP's,~\citet{ref:lamarque2011, ref:vanvuuren2011}). We assume that the sensitivity calculation is independent of the underlying emissions (i.e. that the spatial pattern of the sensitivity of methane to emissions does not change significantly if the precursor emissions change). These emissions serve as an example, as the adjoint sensitivity method could be applied to any (reasonable?) change in emissions.