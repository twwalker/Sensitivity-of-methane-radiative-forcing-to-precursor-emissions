The distribution of the global forcing is weighted toward a small number of locations. At 2050, the top 10\% of locations that have a positive radiative forcing contribution total 49\% of the total positive forcing. The top 10\% of locations with negative radiative forcing contributions total 58\% of the total negative forcing.

The global mean methane radiative forcing due to changes in emissions from 2000 to 2050 is 0.63 uW/m2 for methane, 0.88 uW/m2 for NOx, 0.25 uW/m2 for CO, and 0.03 $\mu $W/m2 for NMHC. Note that these are all positive. Globally, the emissions of methane and NMHC increased, but NOx and CO emissions decreased. The reason a global decrease in CO still contributes a positive radiative forcing is that the places where emissions increased had larger sensitivities, and contributed enough forcing to offset the decreased CO emissions elsewhere.

