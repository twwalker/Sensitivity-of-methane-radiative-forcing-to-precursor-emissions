
The overall change in methane RF at these locations in China is predominantly driven not by changes in direct methane emissions, but by the balance between the magnitude of CO and NOx emissions trends and the resulting overall perturbation to the methane loss rate. Furthermore, the response of the loss rates to precursor trends varies by region, with Shanghai and Beijing being more sensitive to changes in CO and PRD being more sensitive to changes in NOx. 

Future trends of precursor emissions are highly uncertain, and will likely be driven by air quality concerns. In particular, RCP 6.0 is unlikely to provide the correct emissions trajectories. Early-century decadal trends in CO in China suggest that the RCP projections are missing regional declines in CO emissions, particularly in Sichuan, Jiangsu, and Guangdong provinces~\citep{ref:zhao2012}. Evidence of air quality controls on NOx are also seen in satellite observations of decadal trends in NO2 columns from OMI in Beijing, Shanghai, and in the Pearl River Delta~\citep{ref:duncan2016}. However, the decreasing NO2 trends are highly localized to those few urban centers, with large increases dominating in much of the rest of the country. 

%A perturbation study on the RCP emissions in China demonstrates the effect of decreasing the activity levels in the energy, industry, and transportation sectors is decreased by 30\% after 2030 (all species are reduced by this amount). This sectoral decrease represents a decrease of 19-21\% in total NOx, 19-23\% in total CO, 6-7\% in total NMHC, and 3-5\% in total methane at these locations over the century. These changes to the emissions trajectory have almost entirely offsetting forcings in the Pearl River Delta and Sichuan Basin, while the difference in NOx and CO sensitivity in Shanghai and Beijing cause this perturbed emission pathway to have lower total methane radiative forcing than RCP 6.0, by 41\% and 65\% respectively.