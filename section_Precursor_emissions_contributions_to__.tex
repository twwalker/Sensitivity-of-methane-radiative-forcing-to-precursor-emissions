\section{Precursor emissions' contributions to methane radiative forcing in RCP 6.0}

Previous work has calculated global values of the sensitivity of methane lifetimes to precursor emissions~\citep{ref:fry2012,ref:holmes2013}. The spatially explicit sensitivity values calculated with the adjoint model aggregate into a normalized global annual mean sensitivity of the methane lifetime (i.e., $d \mathrm{ln}(\tau)/d\mathrm{ln}(\mathrm{E})$) to land-based anthropogenic NOx, CO, and NMHC emissions of -0.145, 0.047, and 0.027 respectively, highly comparable to previous results.

%However, the method employed here enables the study of regional differences in how methane radiative forcing responds to changes in precursor emissions, and what that implies for the climate impacts of future air quality control and climate change mitigation strategies. As an example, we calculate the methane radiative forcing due to emissions changes projected under representative concentration pathway (RCP) 6.0~\citep{ref:vanvuuren2011}.

Figure~\ref{fig:eqems} shows the methane radiative forcing attributed to changes in NOx, CO, NMHC, and CH4 projected in RCP 6.0 between 2000 and 2100. RCP 6.0, a medium scenario in terms of mitigation efforts, projects roughly a 60\% decrease in global NOx emissions. This decrease, modulated by the high sensitivity of methane radiative forcing to NOx emissions especially in the tropics and subtropics, produces strong positive methane radiative forcing values (decreasing NOx decreases the methane loss rate, leading to a positive forcing). These radiative forcings can be translated into the equivalent change in methane emissions needed to produce the same forcing, which is shown in the right column. We can hence weigh the relative importance of emissions that alter the chemical sink term with the direct source emissions of methane itself. The balance of these two terms (shown in the bottom row) is projected to be a mix of positive and negative forcings (or equivalent emissions) that depend on the local change in each species as well as the sensitivity of the methane loss rate to emissions of said species.

