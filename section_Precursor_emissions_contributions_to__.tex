
Previous work has calculated global values of the sensitivity of methane lifetimes to precursor emissions~\citep{ref:fry2012,ref:holmes2013}. The spatially explicit sensitivity values calculated with the adjoint model aggregate into a normalized global annual mean sensitivity of the methane lifetime (i.e., $d \mathrm{ln}(\tau)/d\mathrm{ln}(\mathrm{E})$) to land-based anthropogenic NOx, CO, and NMHC emissions of -0.145, 0.047, and 0.027 respectively, comparable to results adopted by~\citet{ref:holmes2013} of -0.14 $\pm$ 0.03, 0.06 $\pm$ 0.02, and 0.04 $\pm$ 0.01. Consequently, a doubling of land-based anthropogenic NOx emissions would lead to a methane lifetime reduction of $\approx$ 15\%, which is approximately a factor of 2 higher than a similar combined change in CO and NMHC.  

%However, the method employed here enables the study of regional differences in how methane radiative forcing responds to changes in precursor emissions, and what that implies for the climate impacts of future air quality control and climate change mitigation strategies. As an example, we calculate the methane radiative forcing due to emissions changes projected under representative concentration pathway (RCP) 6.0~\citep{ref:vanvuuren2011}.

These global sensitivities can be spatially attributed to NOx, CO, NMHC, and CH4 emissions according to RCP 6.0 between 2000 and 2050 as shown in Figure~\ref{fig:eqems}. RCP 6.0, a medium scenario in terms of mitigation efforts, projects roughly a 20\% decrease in global NOx emissions by 2050, and up to 60\% at 2100. Tropical and subtropical regional methane RF is amplified by enhanced sensitivity from high surface temperatures, which increase OH production. We also weigh the relative importance of emissions that alter the chemical sink term with the projected direct source emissions of methane itself. The balance of these two terms (shown in the bottom right of Figure~\ref{fig:eqems}) is projected to be a mix of positive and negative forcings that depends on the local change in each species as well as the sensitivity of the methane loss rate to emissions of said species. In particular, Chinese emissions dominate both positive and negative RF from CO+NMHC and NOx emissions, respectively, while playing significant role in direct methane reductions.  The net methane RF is a sensitive balance between these competing terms leading to a complex RF spatial pattern, which will be discussed in more detail in Sec.~\ref{sec:china}.  Similarly, projected increases in direct CH4 RF in the Gangetic plain is offset by negative RF from increased NOx emissions leading to net negative RF in the southern India.  

