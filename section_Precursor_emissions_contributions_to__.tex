\section{Precursor emissions' contributions to methane radiative forcing in RCP 6.0}

Figure~\ref{fig:rfsens} shows the adjoint sensitivity of global methane radiative forcing in July to emissions of various precursor gases. These sensitivities describe the change in methane radiative forcing to a fractional change in precursor emissions at a particular location. Sensitivities to anthropogenic NOx are generally larger at low latitudes in the summer hemisphere, where hydroxyl radical concentrations are highest. Certain locations provide more favorable access for short-lived NOx to reach the global atmosphere, through either high altitude (e.g. Mexico City or Bogot\'a) or strong convection (e.g. southeastern Asia), giving emissions in those locations a stronger influence on the global mean methane loss rate~\citep{ref:bowman2012}. The highest sensitivities do not in general correspond to the locations of largest emissions for NOx, but the tendency for larger values in the tropics could have implications as population growth and economic development move equatorward.