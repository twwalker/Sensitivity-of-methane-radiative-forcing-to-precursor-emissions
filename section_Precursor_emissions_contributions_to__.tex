\section{Precursor emissions' contributions to methane radiative forcing in RCP 6.0}

The spatially explicit sensitivity values (shown in the Methods section) aggregate into a normalized global annual mean sensitivity of the methane lifetime (i.e. $d \mathrm{ln}(\tau)/d\mathrm{ln}(\mathrm{E})$) to land-based anthropogenic NOx, CO, and NMHC emissions of -0.145, 0.047, and 0.027 respectively, comparable to previously reported global values~\citep{ref:fry2012,ref:holmes2013}. However, the method employed here enables the study of regional differences in how methane radiative forcing responds to changes in precursor emissions, and what that implies for the climate impacts of future air quality control and climate change mitigation strategies. As an example, we calculate the methane radiative forcing due to emissions changes projected under representative concentration pathway (RCP) 6.0~\citep{ref:vanvuuren2011}.



% Not explaining the above very well... the idea would be that NOx spends a greater fraction of its lifetime in an urban environment than CO, and the urban environment is less conducive to radical formation. So NOx can have a bigger effect on OH in locations that look like the mean atmosphere while CO has sort of the same influence no matter where it is emitted, so the sensitivity calculation highlights places where a lot of CO is emitted. Does that make sense?

Methane radiative forcing is also sensitive to changes in biomass burning, for example, due to forest management or agricultural practices. While the global annual mean sensitivity of the methane lifetime to biomass burning NOx emissions (-0.029) is about five times lower than that to anthropogenic NOx, the sensitivity to these emissions can be locally dominant, for instance in central Africa.
