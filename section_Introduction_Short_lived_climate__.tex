\section{Introduction}

Short-lived climate forcers alter the Earth's radiative balance on timescales of a decade or less. Pollutants such as NOx, CO, and non-methane hydrocarbons (NMHC) affect the Earth's radiative balance not only through ozone formation, but also alter the local hydroxyl radical concentration, which affects the lifetimes of longer-lived greenhouse gases. Methane loss in particular is dominated by hydroxyl-driven loss in the troposphere~\citep{ref:kirschke2013}.

Accounting of precursor emissions has been lacking in policy discussions, as the provincial/city level effects of these emissions on total radiative forcing (which depends on location) is difficult to calculate efficiently.

Previous studies have calculated the sensitivity of methane lifetime to precursor emissions~\citep{ref:holmes2013}, but lack the ability to calculate these at better than global or regional resolution~\citep{ref:naik2005, ref:macintosh2015}. Full chemistry-climate model studies have been performed for individual megacities~\citep{ref:dang2015}, but the computational cost of perturbing every location individually is prohibitive. Performing this calculation with the adjoint of a chemical transport model provides unprecedented resolution.