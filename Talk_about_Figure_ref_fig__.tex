Figure~\ref{fig:eqemsrank} ranks locations based on the magnitude of the total equivalent emissions calculated above using the 2000-2050 change (left) and the 2000-2100 change (right). At 2050, there is a mixture of positive and negative forcings, while at 2100 the widespread NOx decreases lead primarily to positive values. Three distinct sets of locations can be identified in the 2050 ranking. In the Middle East and India, RCP 6.0 projects substantial increases in methane emissions that drive large values here. In South Africa and Australia, early adoption of NOx controls drives positive values as well. Finally, in China increases in both NOx and CO produce large forcings in opposite directions, the sum of which can produce large totals in either direction (contrast for example Shanghai and the Pearl River Delta).

At 2100, the decreases in NOx emissions are fairly global in extent in the RCP 6.0 projection; the ranking depends more significantly on location and the sensitivity of methane loss at that location. Places in the tropics and the relatively clean Southern Hemisphere rank more highly. Mexico City presents an interesting case where despite large methane cuts, an overall warming is still attributed because of concomitant decreases in NOx. The high altitude of the city contributes to emissions from that location having a particularly large effect on global OH. Similar statements could be made about other larger cities on the list such as Melbourne, Bangkok, and Buenos Aires. The one negative forcing in this ranking (Kota Kinabalu, Malaysia) appears to be due to projected emissions changes brought about by deforestation.