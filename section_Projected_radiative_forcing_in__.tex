\section{Projected radiative forcing in China}

China is projected to more than double both CO and NOx emissions between 2000 and 2050, and 8 of the top 20 equivalent emissions magnitudes at 2050 are located in China. We examine this in more detail, selecting four locations with disparate emissions signatures and differing sensitivity to emissions. Shanghai and Beijing are megacities, and the Pearl River Delta is heavily urbanized, while the Sichuan Basin is an oil and gas producing region in central China. The sensitivity of methane loss to NOx emissions generally decreases with latitude across the four sites (PRD being the largest value), but the two megacities in eastern China have particularly high sensitivity to CO and NMHCs, a feature unique to China. Emissions of NOx, CO, and NMHCs in China are projected to peak around 2050 before declining, with the energy, industry, transportation, and solvent (for NMHC) sectors explaining most of this trend. Methane emissions in China are projected to roughly follow the same trajectory as the global total presented in Figure~\ref{fig:eqems} (top right).

Figure~\ref{fig:intrfchina} shows the century-integrated forcing for the four locations, split by species. Overall, the reduction in the chemical sink caused by high CO emissions outweighs the negative forcings from both high NOx emissions and methane decreases. This is most pronounced at the eastern China sites where the sensitivity of loss rates to those CO emissions is notably high. Despite the high sensitivity to NOx emissions (and consequently large negative forcing) at the lower latitude Pearl River Delta location, high CO emissions in the industry sector there persist longer than at the other locations.