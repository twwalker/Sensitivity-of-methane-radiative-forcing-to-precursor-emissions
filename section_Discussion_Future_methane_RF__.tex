\section{Discussion}

Future methane RF depends not only on the evolution of various methane sources, but also in the spatial distribution of trends in short-lived gases that influence its hydroxyl-driven sink. While future trends in ozone precursors will likely be driven by air quality and energy policies, the chemical impact on methane RF through its sink can be substantial, and either offset or enhance changes in methane sources. The relationship between methane RF and short-lived emissions has significant spatial variability, the quantification of which represents a significant advance in our understanding.

NOx, CO, and NMHC emissions provide dominant controls on ozone production, and this work has ignored the climate effects of precursor emissions changes through the ozone RF, although this impact is discussed in the literature~\citep{ref:naik2005,ref:bowman2012,ref:shindell2013}. The sign of this forcing is generally the opposite of the effect on methane RF for NOx emissions, while it is in the same direction for CO and NMHCs. We have also ignored the additional change in ozone RF caused by a change in methane concentrations. Perturbation studies find the ozone RF response to precursor changes to be larger than that of the methane RF~\citep{ref:akimoto2015}, offsetting it in the case of NOx and enhancing it in the case of CO and NMHCs. We have also not included the effects of precursor emissions on sulfate burdens or aerosol RF. However, in regional studies the magnitude of the response of aerosol RF is smaller than that of methane~\citep{ref:fry2012}. Nevertheless, the study of the effects of ozone precursors on methane radiative forcing and their explicit spatial distribution provides a crucial building block for future work that contrasts these competing forcing terms. The consistent calculation of the effects of historic changes in grid-scale NOx and CO emissions on both ozone and methane radiative forcing using an adjoint methodology is deferred to a future study.

The mitigation of both direct methane and precursor emissions will be an important component of any comprehensive climate policy.  However, emission changes and their radiative impact must be determined globally at relatively fine scales (< 200$\times$ 200 km$^2$) in order to quantify the efficacy of these policies.  A new satellite observing system of methane and precursor concentrations in conjunction with advanced assimilation techniques could serve as a key pillar of such policies~\cite{Bowman:2013kx}. The methodology described here can be readily adapted to this observing system as a means of monitoring the trajectory of emissions and their impact on methane RF.  

%What follows are the many caveats to interpreting these results:

%We have assumed that the sensitivity calculation is independent of the underlying emissions. This does make the calculation of radiative forcing from an arbitrary change in emissions possible.

%We have assumed that the system can be approximated by a linearized model around the present-day atmospheric chemical state.

%We have assumed the parameterizations of natural sources of NOx and CO (biomass burning, lightning) contained in GEOS-Chem are sufficiently accurate to not bias the calculation of the sensitivity to anthropogenic emissions.

%We have calculated the annual mean sensitivity, although the sensitivity in different months does vary. This could have implications for shifts in energy demand that favor a particular season, such as heating or air conditioning loads.

%We have calculated the global methane radiative forcing, not a regional forcing.
