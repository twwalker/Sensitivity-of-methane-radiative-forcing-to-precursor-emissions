In order to compare the radiative impact between different precursor emissions, we translate the methane RF from precursors into the equivalent change in methane emissions required to produce a the same RF. Figure~\ref{fig:eqemsrank} ranks locations based on the magnitude of the total equivalent CH4 emissions using changes in emissions from 2000-2050 (left) and from 2000-2100 (right). Over both time periods, the emission  mixture leads to both  positive and negative equivalent emissions. For example, Shanghai has positive equivalent emissions due to a CO emissions increase, but there are also negative equivalent emissions due to a NOx emissions increase and a CH4 emissions reductions. The total, represented by the cross

Three distinct sets of locations can be identified in the 2050 ranking. In the Middle East and India, RCP 6.0 projects substantial increases in direct methane emissions. Early adoption of NOx controls in South Africa, Australia, and North America leads to positive equivalent CH4 emissions. In China, increases in both NOx and CO produce opposing equivalent CH4 emissions, which are dominated by  CO emissions.

At 2100, chemical loss dominates over direct CH4 emissions at many locations, driven primarily by fairly uniform decreases in NOx emissions where the ranking depends more significantly on location. For example,  relatively clean Southern Hemisphere locations are ranked more highly even though the absoluted emission changes are smaller. The top ranked location over this time scale is in northern Canada, where a large decrease in NOx emissions is projected. Exceptions to the dominance of NOx decreases appear in northern hemisphere cities where both methane and CO decreases outweigh the impact of NOx decreases, such as Moscow, Toronto, and Chicago. 

%Mexico City presents an interesting case where despite large methane cuts, an overall warming is still attributed because of concomitant decreases in NOx. The high altitude of the city contributes to precursor emissions from that location having a particularly large effect on global OH. As can be seen from the adjoint sensitivities (see Methods), the RF per unit change in NOx emissions over Mexico City is a factor of four times greater than the North American average.  Similar statements could be made about other larger cities on the list such as Melbourne, Bangkok, and Buenos Aires. The one negative value in this ranking (Kota Kinabalu, Malaysia) is due to projected NOx emissions increases in RCP 6.0 in the deforestation and land use change sector.