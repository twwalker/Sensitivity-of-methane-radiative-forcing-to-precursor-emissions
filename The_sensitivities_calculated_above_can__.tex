The sensitivities calculated above can be applied to (reasonably) arbitrary changes in precursor emissions to investigate the effect of projected emissions changes on methane radiative forcing. Figure~\ref{fig:drfcomb85} shows the change in radiative forcing under emissions changes predicted by RCP 8.5 between 2000 and 2050. In this scenario, methane emissions increase globally and lead to positive radiative forcing contributions everywhere. NOx emissions continue to decrease in North America and Europe, but increase in the first half of the century in developing India and China. CO emissions rise in India and generally decrease elsewhere. The effects of these changes in NOx, CO, and CH4 emissions on the global methane radiative forcing are shown separately, and represent a combination of the emissions changes, given by the RCP, and the local sensitivity of the methane radiative forcing to these changes, given by the adjoint analysis. Note that the sensitivity of direct methane emissions to its radiative forcing is assumed to be the same everywhere, as it is a globally well-mixed gas.