One major advantage of having spatially explicit attribution of the radiative forcing to precursor emissions is that it allows for quick comparison among regions of interest. For example, Table~\ref{tab:megacity} shows the sensitivity of methane lifetime to precursor emissions averaged over $3\times 3$ grid cell areas centered on the megacities recently studied in fully coupled chemistry-climate simulations~\citep{ref:dang2015}. These values are not directly comparable to the radiative forcing due to a one year pulse of all emissions from each of these megacities; however, they provide some insights into the spatial variability of the effects of precursor emissions on climate.

All the megacity sites except Lagos demonstrate sensitivity higher in magnitude than the global mean values for all precursor emissions.

\begin{table}
  \begin{tabular}{ c c c c }
        City & $d \mathrm{ln}(\tau)/d \mathrm{ln}(\mathrm{E_{NOx}})$ & $d \mathrm{ln}(\tau)/d \mathrm{ln}(\mathrm{E_{CO}})$ & $d \mathrm{ln}(\tau)/d \mathrm{ln}(\mathrm{E_{NMHC}})$ \\ 
        Los Angeles  & -0.68 & 0.24 & 0.09 \\ 
        New York     & -0.91 & 0.49 & 0.30 \\ 
        Mexico City  & -0.71 & 0.15 & 0.11 \\ 
        Sao Paulo    & -0.50 & 0.05 & 0.07 \\ 
        Lagos        & -0.15 & 0.03 & 0.05 \\ 
        Cairo        & -0.19 & 0.02 & 0.02 \\ 
        New Delhi    & -0.62 & 0.40 & 0.18 \\ 
        Beijing      & -1.30 & 2.13 & 0.02 \\ 
        Shanghai     & -0.45 & 1.04 & 0.05 \\ 
        Manila       & -0.28 & 0.03 & 0.04 \\ 
    \end{tabular} 
    \label{tab:megacity}
    \caption{Fractional sensitivity of global methane lifetime to emissions from various megacity regions.} 
\end{table}

