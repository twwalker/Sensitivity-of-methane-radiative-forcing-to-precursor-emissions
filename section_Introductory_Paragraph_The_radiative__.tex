\section*{Introductory Paragraph}

The radiative forcing (RF) from methane is a consequence of both direct emissions of methane and chemical loss from emissions of short-lived precursor gases NOx, CO, and non-methane hydrocarbons (NMHCs). Dramatic changes in the spatial distribution of these emissions and their uncertain trajectory have important implications for both climate forcing and air quality mitigation.  We attribute here net methane RF  to spatially-resolved emissions using an adjoint methodology and compute the impact of these emissions on future radiative forcing for the Representative Concentration Pathway (RCP) 6.0.  Normalized methane lifetime sensitivity to annual anthropogenic NOx emissions is -0.15 whereas to CO and NMHC emissions it is 0.05 and 0.03, respectively. The net RF from the opposing effects of NOx and combined CO and NMHC emissions varies considerably depending upon the location and magnitude of the emissions.  Under RCP 6.0 in 2050,  we find the combined top 10\% of locations with positive net RF is responsible for 50\% of the global positive RF and the top 10\% of locations with negative RF cause 60\% of the global negative RF. As the largest source of current emissions, we examine in greater detail the regional Chinese contribution to methane RF focusing on the Pearl River Delta, Shanghai, Sichuan Basin and Beijing. Effects of CO emissions dominate over both methane and NOx emissions. However, the response of these emission trajectories to near-term air quality and energy policies will likely alter the emission mixture and consequently net methane RF. This methodology will be able to relate future changes in these emissions--potentially inferred from satellite observations--to net methane RF  as part of a global climate and air quality monitoring system. 