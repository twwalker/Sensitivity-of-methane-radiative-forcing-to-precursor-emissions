\section*{Introductory Paragraph}

The radiative forcing (RF) from methane is a consequence of both direct emissions of methane and chemical loss from emissions of short-lived precursor gases NOx, CO, and non-methane hydrocarbons (NMHCs). Dramatic changes in the spatial distribution of these emissions and their uncertain trajectory have important implications for both climate forcing and air quality mitigation.  We attribute here methane RF to spatially-resolved emissions using an adjoint methodology and compute the impact of emissions under the Representative Concentration Pathway (RCP) 6.0 on future methane RF. The sensitivity of the methane sink to the opposing effects of NOx and combined CO and NMHC emissions varies considerably depending upon the location and magnitude of the emissions changes. Under RCP 6.0 in 2050, we find the top 10\% of locations with positive methane RF are responsible for 50\% of the global positive methane RF and the top 10\% of locations with negative methane RF cause 60\% of the global negative methane RF. We examine in greater detail regional variations in the sensitivity of methane RF to precursor emissions changes in China, focusing on the Pearl River Delta, Shanghai, the Sichuan Basin, and Beijing. Effects of CO emissions dominate over both methane and NOx emissions. However, the response of these emission trajectories to near-term air quality and energy policies will likely alter the emission mixture and consequently net methane RF. This methodology will be able to relate future changes in these emissions to net methane RF as part of a global climate and air quality monitoring system. 