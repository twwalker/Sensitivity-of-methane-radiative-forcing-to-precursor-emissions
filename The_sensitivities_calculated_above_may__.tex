The sensitivities calculated above may also be applied to (reasonably) arbitrary changes in precursor emissions to investigate the effect of projected emissions changes on methane radiative forcing. Figure~\ref{fig:drfcomb85} shows the change in radiative forcing under emissions changes predicted by RCP 8.5 between 2000 and 2050. In this scenario, methane emissions increase globally and lead to positive radiative forcing contributions everywhere. Note that the sensitivity of direct methane emissions to its radiative forcing is assumed to be the same everywhere, as it is a globally well-mixed gas. The contribution of NOx to methane radiative forcing is positive across North America and Europe where controls are in place and emissions are expected to continue declining. Increased activity in the developing world leads to increased NOx emissions (and negative methane radiative forcing) under this scenario, particularly in India and southern China, but also in the Middle East, Africa, and South America. CO emissions rise and contribute a positive methane radiative forcing in India, and generally decrease elsewhere. The effects of these changes in NOx, CO, and CH4 emissions on the global methane radiative forcing are shown separately, and represent a combination of the emissions changes, given by the RCP, and the local sensitivity of the methane radiative forcing to these changes, given by the adjoint analysis. 

% Going to need some total values for specific continents or locations