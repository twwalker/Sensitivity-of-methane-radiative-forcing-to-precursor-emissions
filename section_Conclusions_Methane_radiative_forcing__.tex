\section{Conclusions}

Methane radiative forcing is sensitive not only to changes in various methane sources, but also in the spatial distribution of short-lived gases that influence its sink, hydroxyl-driven loss.

We have assumed that the sensitivity calculation is independent of the underlying emissions. This does make the calculation of radiative forcing from an arbitrary change in emissions possible.

We have assumed that the system can be approximated by a linearized model around the present-day atmospheric chemical state.

We have not included the effects of NOx and CO emissions on ozone radiative forcing. The sign of this forcing is generally the opposite of the effect on methane radiative forcing for NOx emissions. The calculation of the effect of changes in NOx and CO emissions on ozone and methane radiative forcing combined during the Aura satellite period, as constrained by global satellite observations of NO2, CO, and O3, is deferred to a future study. We have also ignored the additional change in ozone radiative forcing caused by a change in methane concentrations.

We have not included the effects of NOx and CO emissions on sulfate burden or aerosol radiative forcing. This is smaller than the methane radiative forcing (e.g. Fry 2012).

We have assumed the parameterizations of natural sources of NOx and CO (biomass burning, lightning) contained in GEOS-Chem are sufficiently accurate to not bias the calculation of the sensitivity to anthropogenic emissions.

We have calculated the annual mean sensitivity, although the sensitivity in different months does vary.

We have calculated the global methane radiative forcing, not a regional forcing.
