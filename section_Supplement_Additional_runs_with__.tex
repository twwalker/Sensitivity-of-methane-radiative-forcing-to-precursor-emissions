\section{Supplement}

Additional runs with the GEOS-Chem adjoint model were performed to test the robustness of the result to the choice of simulation time used to calculate the sensitivities, and to the methodology's assumption of linearity over the range of emissions being tested. Simulations over 30-day and 60-day time windows were performed and the results are contrasted to the 75-day simulations in Figure~\ref{fig:diffwindow}. In these runs, we calculate the sensitivity of the methane lost over the full time period to the emissions during the first 30 days. The percent difference between the 30- and 60-day simulations is large, as there remains some influence of the emissions perturbation at the end of the 30-day simulation that is accounted for in the longer simulation. However, the difference between 60- and 75-day simulations becomes small, indicating that this time window is sufficiently long, even for the longer-lived emitted tracers such as CO, to account for the entire influence of an emissions perturbation in the first 30 days.
