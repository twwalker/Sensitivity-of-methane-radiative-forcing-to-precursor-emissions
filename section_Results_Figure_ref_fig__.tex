\section{Results}

Figure~\ref{fig:drfdenox} shows the annual mean sensitivity of global methane radiative forcing to anthropogenic emissions of NOx. The negative sign implies that increases in NOx emissions lower the global methane radiative forcing, because higher emissions locally decrease hydroxyl radical concentrations, reducing the loss of methane. The sensitivities are largest at low latitudes where hydroxyl radical concentrations are highest, and in general do not map to the locations of largest emissions. Certain locations provide more favourable access to the global atmosphere through high altitude (e.g. Mexico City) or strong convection (e.g. southeastern Asia), giving emissions in those locations a stronger influence on the global mean methane loss rate~\citep{ref:bowman2012}.

Sensitivities with respect to anthropogenic NOx are greater than those with respect to CO or VOCs. However, methane loss rates are also highly sensitive to lightning NOx emissions.

% Need to find a metric to compare overall 'strength' of sensitivity