\section{Sensitivity of methane radiative forcing to precursor emissions}

Figure~\ref{fig:rfsens} shows the sensitivity of global methane radiative forcing in July to emissions of various short-lived precursor gases. These sensitivities reflect the amount of change in the global methane radiative forcing due to a fractional change in precursor emissions at a particular location. These values correspond to a mean global annual sensitivity of the methane lifetime ($d \mathrm{ln}(\tao)/d \mathrm{ln}(\mathrm{E})$) of -0.145 for anthropogenic NOx, 0.047 for anthropogenic CO, and 0.027 for NMHC's, similar to previous studies at the global scale~\citep{ref:fry2012,ref:holmes2013}.

The negative sign for NOx sensitivity implies that increases in NOx emissions lower the global methane radiative forcing, because higher emissions locally decrease hydroxyl radical concentrations, reducing the loss of methane, while the inverse is generally true for CO and NMHC emissions. The sensitivities are generally largest at low latitudes where hydroxyl radical concentrations are highest. Certain locations provide more favorable access to the global atmosphere through high altitude (e.g. Mexico City) or strong convection (e.g. southeastern Asia), giving emissions in those locations a stronger influence on the global mean methane loss rate~\citep{ref:bowman2012}.

High sensitivities in general do not map to the locations of largest emissions for NOx, but for CO and some of the longer-lived VOCs the highest sensitivities do occur in strong source regions. Why would this be?

Sensitivities with respect to anthropogenic NOx are greater than those with respect to CO or VOCs. However, methane loss rates are also highly sensitive to biomass burning and lightning NOx emissions and these can be locally dominant, for instance in central Africa.

Methane radiative forcing is about ten times more sensitive to changes in anthropogenic NOx than to anthropogenic NMHC's (here, for the purposes of the speciated NMHC's in the GEOS-Chem mechanism, the combination of ALK4, ALD2, MEK, PRPE, C3H8, C2H6, and CH2O... differences between the relative weight of each species between GEOS-Chem and the RCP scenarios are neglected). This is partly because some VOCs (CH2O in particular) can contribute a negative forcing in certain chemical environments which offsets the general increase of OH with increasing VOC emissions. The VOC sensitivity can be comparable to NOx in some polluted regions, for instance the Ganges valley.

% Need to find a metric to compare overall 'strength' of sensitivity