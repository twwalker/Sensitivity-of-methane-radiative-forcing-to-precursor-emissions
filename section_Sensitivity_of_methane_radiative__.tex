\section{Sensitivity of methane radiative forcing to precursor emissions}

Figure~\ref{fig:rfsens} shows the sensitivity of global methane radiative forcing in July to emissions of various precursor gases. These sensitivities describe the change in methane radiative forcing to a fractional change in precursor emissions at a particular location. These spatially explicit values aggregate into a normalized global annual mean sensitivity of the methane lifetime (i.e. $d \mathrm{ln}(\tau)/d\mathrm{ln}(\mathrm{E})$) of -0.145 for anthropogenic NOx, 0.047 for anthropogenic CO, and 0.027 for anthropogenic NMHC's, comparable to previous global-scale studies~\citep{ref:fry2012,ref:holmes2013}. The method employed here, however, enables the study of regional differences in this sensitivity, and what that implies for the climate impacts of future air quality control strategies.

%The negative sign for NOx sensitivity implies that increases in NOx emissions lower the global methane radiative forcing, because higher emissions locally increase hydroxyl radical concentrations, increasing the loss of methane and thereby producing a negative forcing, while the inverse is true for CO emissions. 

Sensitivities to anthropogenic NOx are generally largest at low latitudes in the summer hemisphere, where hydroxyl radical concentrations are highest. Certain locations provide more favorable access for short-lived NOx to reach the global atmosphere, through either high altitude (e.g. Mexico City, Bogot\'a) or strong convection (e.g. southeastern Asia), giving emissions in those locations a stronger influence on the global mean methane loss rate~\citep{ref:bowman2012}. The highest sensitivities do not in general correspond to the locations of largest emissions for NOx, but the larger values in the tropics could have implications as population growth and economic development move equatorward.

%, but for CO and some of the longer-lived VOCs the highest sensitivities do occur in strong source regions. Why would this be?

The global mean sensitivity of methane radiative forcing with respect to anthropogenic NOx is greater than that with respect to CO or NMHC's. However, because of their longer lifetimes, CO and NMHC's are better mixed than NOx so the sensitivities map more directly to locations of high emissions. In particular, CO emissions in eastern China exhibit high sensitivities, as do NMHC emissions in the Indo-Gangetic plain. The overall change in methane radiative forcing depends on the local changes in each precursor, modulated by the sensitivity of each type of emissions change in that location.

Methane radiative forcing is also sensitive to changes in biomass burning, for example, due to forest management or agricultural practices. While the global annual mean sensitivity of the methane lifetime to biomass burning NOx emissions (-0.029) is about five times lower than that of anthropogenic NOx, the sensitivity to these emissions can be locally dominant, for instance in central Africa.

%Methane radiative forcing is about ten times more sensitive to changes in anthropogenic NOx than to anthropogenic VOCs (here, for the purposes of the speciated VOCs in the GEOS-Chem mechanism, the combination of ALK4, ALD2, MEK, PRPE, C3H8, C2H6, and CH2O... differences between the relative weight of each species between GEOS-Chem and the RCP scenarios are neglected). This is partly because some VOCs (CH2O in particular) can contribute a negative forcing in certain chemical environments which offsets the general increase of OH with increasing VOC emissions. The VOC sensitivity can be comparable to NOx in some polluted regions, for instance the Ganges valley.

% Need to find a metric to compare overall 'strength' of sensitivity