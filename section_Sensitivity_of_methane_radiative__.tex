\section{Sensitivity of methane radiative forcing to precursor emissions}

Figure~\ref{fig:rfsens} shows the sensitivity of global methane radiative forcing in July to emissions of various precursor gases. These sensitivities describe the change in methane radiative forcing to a fractional change in precursor emissions at a particular location. The spatially explicit values shown in Figure~\ref{fig:rfsens} aggregate into a normalized global annual mean sensitivity of the methane lifetime (i.e. $d \mathrm{ln}(\tau)/d\mathrm{ln}(\mathrm{E})$) to land-based anthropogenic NOx, CO, and NMHC emissions of -0.145, 0.047, and 0.027 respectively, comparable to previously reported global values~\citep{ref:fry2012,ref:holmes2013}. However, the method employed here enables the study of regional differences in how methane radiative forcing responds to changes in precursor emissions, and what that implies for the climate impacts of future air quality control and climate change mitigation strategies.

%The negative sign for NOx sensitivity implies that increases in NOx emissions lower the global methane radiative forcing, because higher emissions locally increase hydroxyl radical concentrations, increasing the loss of methane and thereby producing a negative forcing, while the inverse is true for CO and (most) NMHC emissions. 

Sensitivities to anthropogenic NOx are generally larger at low latitudes in the summer hemisphere, where hydroxyl radical concentrations are highest. Certain locations provide more favorable access for short-lived NOx to reach the global atmosphere, through either high altitude (e.g. Mexico City, Bogot\'a) or strong convection (e.g. southeastern Asia), giving emissions in those locations a stronger influence on the global mean methane loss rate~\citep{ref:bowman2012}. The highest sensitivities do not in general correspond to the locations of largest emissions for NOx, but the tendency for larger values in the tropics could have implications as population growth and economic development move equatorward.

On a global scale, the mean sensitivity of methane radiative forcing with respect to anthropogenic NOx is greater than that with respect to CO or NMHC's. However, because of their longer atmospheric lifetimes relative to NOx, CO and NMHC emissions are more exposed to the free troposphere, so the sensitivities for these precursors map more directly to locations of high emissions. In particular, CO emissions in eastern China exhibit high sensitivities, as do NMHC emissions in the Indo-Gangetic plain. The overall change in methane radiative forcing depends on the local changes in each precursor, modulated by the sensitivity of each type of emissions change in that location.

% Not explaining the above very well... the idea would be that NOx spends a greater fraction of its lifetime in an urban environment than CO, and the urban environment is less conducive to radical formation. So NOx can have a bigger effect on OH in locations that look like the mean atmosphere while CO has sort of the same influence no matter where it is emitted, so the sensitivity calculation highlights places where a lot of CO is emitted. Does that make sense?

Methane radiative forcing is also sensitive to changes in biomass burning, for example, due to forest management or agricultural practices. While the global annual mean sensitivity of the methane lifetime to biomass burning NOx emissions (-0.029) is about five times lower than that of anthropogenic NOx, the sensitivity to these emissions can be locally dominant, for instance in central Africa.
