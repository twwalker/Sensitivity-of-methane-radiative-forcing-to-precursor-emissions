In order to more easily compare the impact between different precursor emissions, we translate the methane RF into the equivalent change in methane emissions required to produce an equivalent RF. Figure~\ref{fig:eqemsrank} ranks locations based on the magnitude of the total equivalent CH4 emissions using changes in emissions from 2000-2050 (left) and from 2000-2100 (right). At 2050, there is a mixture of positive and negative forcings, while at 2100 the widespread NOx decreases lead primarily to positive values. Three distinct sets of locations can be identified in the 2050 ranking. In the Middle East and India, RCP 6.0 projects substantial increases in direct methane emissions  while early adoption of NOx controls in South Africa and Australia lead to equivalent positive emissions. In China, however, increases in both NOx and CO produce  equivalent emissions in opposite directions, the sum of which can produce substantial totals in either direction e.g., Shanghai and the Pearl River Delta.

At 2100, chemical loss dominates over direct CH4 emissions driven primarily by  fairly uniform decreases in NOx emissions.  Consequently, the ranking depends more significantly on  the sensitivity of methane loss to NOx changes at a location. Locations in the tropics and the relatively clean Southern Hemisphere rank more highly. Mexico City presents an interesting case where despite large methane cuts, an overall warming is still attributed because of concomitant decreases in NOx. The high altitude of the city contributes to precursor emissions from that location having a particularly large effect on global OH. As can be seen from the adjoint sensitivities (see Methods), the RF per unit change in NOx emissions over Mexico City is a factor of four times greater than the North American average. Similar statements could be made about other larger cities on the list such as Melbourne, Bangkok, and Buenos Aires. The one negative value in this ranking (Kota Kinabalu, Malaysia) appears to be due to projected NOx emissions increases in RCP 6.0 brought about by deforestation.