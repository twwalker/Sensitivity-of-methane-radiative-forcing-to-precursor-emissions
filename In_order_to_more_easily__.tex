In order to more easily compare the impact between precursor emissions, we translate the methane RF from an emissions into the equivalent change in methane emissions required to produce an equivalent RF. Figure~\ref{fig:eqemsrank} ranks locations based on the magnitude of the total equivalent emissions using changes in emissions from 2000-2050 (left) and from 2000-2100 (right). At 2050, there is a mixture of positive and negative forcings, while at 2100 the widespread NOx decreases lead primarily to positive values. Three distinct sets of locations can be identified in the 2050 ranking. In the Middle East and India, RCP 6.0 projects substantial increases in methane emissions that drive large positive equivalent emissions here. In South Africa and Australia, early adoption of NOx controls drives positive values as well. Finally, in China increases in both NOx and CO produce large values of equivalent emissions in opposite directions, the sum of which can produce substantial totals in either direction (contrast, for example, Shanghai and the Pearl River Delta).

At 2100, the decreases in NOx emissions are fairly global in extent in the RCP 6.0 projection; the ranking depends more significantly on location and the sensitivity of methane loss to NOx changes at that location. Locations in the tropics and the relatively clean Southern Hemisphere rank more highly. Mexico City presents an interesting case where despite large methane cuts, an overall warming is still attributed because of concomitant decreases in NOx. The high altitude of the city contributes to precursor emissions from that location having a particularly large effect on global OH. Similar statements could be made about other larger cities on the list such as Melbourne, Bangkok, and Buenos Aires. The one negative value in this ranking (Kota Kinabalu, Malaysia) appears to be due to projected NOx emissions increases brought about by deforestation.