\section{Methane radiative forcing attributed to projected precursor emissions in China}
\label{sec:china}
China is projected to more than double both CO and NOx emissions between 2000 and 2050, and 12 of the top 30 equivalent emissions magnitudes at 2050 are located in China (see Figure~\ref{fig:eqemsrank}). Furthermore, 50\% of the positive equivalent emissions attributable to changes in CO occurs in the top 5\% locations with positive values, all of which are located in eastern China. Increases in CO emissions are especially significant, since the effects of CO emissions on methane RF and ozone RF have the same sign whereas they are of opposite sign in the case of NOx~\citep{ref:myhre2013}. The sensitivity of global methane loss rates to precursor emissions in China also has substantial gradients -- sensitivity to NOx varies by a factor of two between $20^{\circ}$N and $45^{\circ}$N over China, and sensitivity to CO by a factor of seven between $120^{\circ}$E and $90^{\circ}$E -- indicating that the impact of emissions controls on methane RF will vary by location. 

We examine changes in China in more detail, selecting four locations with disparate emissions signatures and differing sensitivities to emissions. Shanghai and Beijing are megacities, and the Pearl River Delta (PRD) is heavily urbanized, while the Sichuan Basin is an oil and gas producing region in central China. The sensitivity of methane loss to NOx emissions generally decreases with latitude across the four sites (PRD has a 25\% larger sensitivity to NOx than Beijing), but the two megacities in eastern China have particularly high sensitivity to CO, a feature unique to China. Specifically, the sensitivity of  emissions changes in Shanghai is a factor of three greater than that in PRD. Under RCP 6.0, emissions of NOx, CO, and NMHCs in China are projected to peak around 2050 before declining, with the energy, industry, transportation, and solvent (for NMHC) sectors explaining most of this behaviour. Methane emissions in China are projected to follow the same trajectory as the global total in RCP 6.0, with a 10\% relative increase at 2060 followed by a decline to 20\% below year 2000 values by the end of the century.

Figure~\ref{fig:intrfchina} shows the century-integrated methane RF for the four locations, split by species. Overall, the reduction in the chemical sink caused by high CO emissions outweighs the negative forcings from both increases in NOx emissions and direct methane emission decreases. This is most pronounced at the eastern China sites where the sensitivity of methane loss rates to CO emissions is notably high. Despite the high sensitivity to NOx emissions (and consequently large negative forcing) at the lower latitude PRD location, high CO emissions in the industrial sector there persist to the end of the century, longer than at the other locations, and contribute a greater integrated forcing.