\section{Discussion}

Future methane radiative forcing depends not only on the evolution of various methane sources, but also in the spatial distribution of trends in short-lived gases that influence its sink, hydroxyl-driven loss. While future trends in ozone precursors will likely be driven by air quality policies, the chemical impact on methane radiative forcing through its sink can be substantial, and either offset or enhance changes in methane sources.

NOx, CO, and NMHC emissions provide a dominant control on ozone production, and this work has ignored the climate effects of precursor emissions changes through the ozone radiative forcing, although this impact is discussed in the literature~\citep{ref:shindell2013}. The sign of this forcing is generally the opposite of the effect on methane radiative forcing for NOx emissions, while it is in the same direction for CO and NMHCs. We have also ignored the additional change in ozone radiative forcing caused by a change in methane concentrations. Perturbation studies find the ozone radiative forcing response to precursor changes to be larger than that of the methane radiative forcing~\citep{ref:akimoto2015}, offsetting it in the case of NOx and enhancing it in the case of CO and NMHCs. We have also not included the effects of precursor emissions on sulfate burdens or aerosol radiative forcing. However, in regional studies the magnitude of the response of aerosol radiative forcing is smaller than that of methane radiative forcing~\citep{ref:fry2012}.

The consistent calculation of the effects of past changes in NOx and CO emissions on both ozone and methane radiative forcing during the Aura satellite period, as constrained by global satellite observations of NO2, CO, and O3, is deferred to a future study.

%What follows are the many caveats to interpreting these results:

%We have assumed that the sensitivity calculation is independent of the underlying emissions. This does make the calculation of radiative forcing from an arbitrary change in emissions possible.

%We have assumed that the system can be approximated by a linearized model around the present-day atmospheric chemical state.

%We have assumed the parameterizations of natural sources of NOx and CO (biomass burning, lightning) contained in GEOS-Chem are sufficiently accurate to not bias the calculation of the sensitivity to anthropogenic emissions.

%We have calculated the annual mean sensitivity, although the sensitivity in different months does vary. This could have implications for shifts in energy demand that favor a particular season, such as heating or air conditioning loads.

%We have calculated the global methane radiative forcing, not a regional forcing.
