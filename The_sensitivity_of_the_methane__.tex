The sensitivity of the methane radiative forcing to $\mathrm{NO_x}$ emissions is location-dependent, and differs from the pattern of maximum emissions. The sensitivities are stronger in the tropics than at midlatitudes because hydroxyl abundances are greater in the tropics. Also, sensitivities are larger in locations where emissions are rapidly dispersed and can have a greater effect on the global hydroxyl abundance (e.g. Mexico City at high altitude, or Kuala Lumpur which is subject to strong convection).

Figure~\ref{fig:drfcomb85} shows the change in radiative forcing under emissions changes predicted by RCP 8.5 in the year 2050. In this scenario, methane emissions increase globally, while NOx continues to decrease in North America and Europe, but increases in the first half of the century in developing India and China. This increase in NOx produces a negative radiative forcing response in methane in those regions, which competes with the positive forcing from methane emissions increases.