
On a global scale, the mean sensitivity of methane radiative forcing with respect to anthropogenic NOx is greater than that with respect to CO or NMHC's. However, because of their longer atmospheric lifetimes relative to NOx, CO and NMHC emissions are more exposed to the free troposphere, so the sensitivities for these precursors map more directly to locations of high emissions. In particular, CO emissions in eastern China exhibit high sensitivities, as do NMHC emissions in the Indo-Gangetic plain. The overall change in methane radiative forcing depends on the local changes in each precursor, modulated by the sensitivity of each type of emissions change in that location.

% Not explaining the above very well... the idea would be that NOx spends a greater fraction of its lifetime in an urban environment than CO, and the urban environment is less conducive to radical formation. So NOx can have a bigger effect on OH in locations that look like the mean atmosphere while CO has sort of the same influence no matter where it is emitted, so the sensitivity calculation highlights places where a lot of CO is emitted. Does that make sense?

Methane radiative forcing is also sensitive to changes in biomass burning, for example, due to forest management or agricultural practices. While the global annual mean sensitivity of the methane lifetime to biomass burning NOx emissions (-0.029) is about five times lower than that to anthropogenic NOx, the sensitivity to these emissions can be locally dominant, for instance in central Africa. We make the simplifying assumption that precursor sources other than direct anthropogenic emissions have no long-term trend, and that the estimate of these emissions in GEOS-Chem is sufficiently accurate to not bias the sensitivity calculation.
