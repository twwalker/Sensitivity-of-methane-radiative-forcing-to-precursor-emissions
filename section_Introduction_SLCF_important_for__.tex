\section{Introduction}

% SLCF - important for climate
%  - methane is second most important contributor to RF (26% of CO2 forcing)*
%  - ozone is third most important (22% of CO2 forcing)*
%  * according to AR5, here ref:myhre2013

% Precursors of ozone formation
%  - climate, air quality co-benefit
%  - affect ozone formation, also alter local hydroxyl radical concentrations
%  - methane loss in particular is dominated by hydroxyl driven loss in troposphere

Methane ranks as the second most significant anthropogenic greenhouse gas, with 26\% the preindustrial to present day forcing of CO2, and its atmospheric burden continues to increase~\citep{ref:myhre2013}. Chemical oxidation by the hydroxyl radical (OH) in the troposphere accounts for up to 90\% of the total sink of atmospheric methane~\citep{ref:kirschke2013}. While methane is well-mixed in the atmosphere, many of the compounds that alter OH concentrations have short atmospheric lifetimes, and can cause local variations in the rate of hydroxyl-driven methane loss. A number of these short-lived gases, such as NOx, CO, and non-methane hydrocarbons (NMHCs), also have significant effects on air quality due to their strong control on ozone and particulate matter pollution~\citep{ref:driscoll2015}. The health impact of emissions of ozone precursor gases, accentuated by recent catastrophic air quality events, are likely to shape trends in these short-lived gases in the coming decades.

%Short-lived climate forcers alter the Earth's radiative balance on timescales of a decade or less. Pollutants such as NOx, CO, and non-methane hydrocarbons (NMHC's) have significant air quality co-benefits due to their strong control on ozone and particulate matter pollution~\citep{ref:driscoll2015}. These precursor gases affect the Earth's radiative balance not only through ozone formation, but also alter the local hydroxyl radical concentration, which affects the lifetimes of longer-lived greenhouse gases. Methane loss in particular is dominated by hydroxyl-driven loss in the troposphere, accounting for up to 90\% of the total sink~\citep{ref:kirschke2013}. Methane is the second most significant anthropogenically-produced contributor to radiative forcing, with about 26\% the preindustrial to present day forcing of CO2~\citep{ref:myhre2013}.

The climate impact of changing emissions of ozone precursors through their effect on tropospheric ozone concentrations has been studied previously~\citep{ref:shindell2013}. However, accounting for the relation between these emissions, local methane loss rates, and global methane abundances has been limited to studies at global~\citep{ref:prather2001,ref:fry2012,ref:holmes2013} or continental scales~\citep{ref:naik2005,ref:macintosh2015}. The response of hydroxyl concentrations to changes in precursor emissions depends critically on the local chemical environment, and perturbations averaged over large scales are not reflective of the environment within urban centres where emissions changes are likely to be most pronounced. Hydroxyl concentrations in the ACCMIP chemistry climate simulations responded at regional scales to projected changes in ozone precursor emissions~\citep{ref:voulgarakis2013}, although attribution to a specific precursor or location was not possible. Full chemistry-climate model studies have been performed for individual megacities~\citep{ref:dang2015}, but the computational cost of perturbing every location individually remains prohibitive. 

% OH responds to many factors
%  - primarily methane concentraions, but these vary slowly
%  - ozone precursors alter local hydroxyl chemistry
%  - changes in photolysis rates through stratospheric ozone burden, clouds, aerosols not covered here

%Hydroxyl concentrations in the ACCMIP chemistry climate simulations~\citep{ref:voulgarakis2013} responded primarily to changes in methane concentrations, but also to changes in NOx, NMHCs, and photolysis rates. 

%Small variations in OH abundance help explain interannual variations in the growth rates of global methane concentrations, including its recent hiatus~\citep{ref:mcnorton2016} \textbf{KB: not sure if they'll accept an ACPD ref}. Importantly, variability in OH concentrations is more significant than changes in temperature in explaining methane growth rates \textbf{KB: ref?}.

% effect of NOx, CO, NMHC on CH4 lifetimes only known at coarse resolution
%  - calculation requires fully resolved tropospheric chemistry simulation (expensive)
%  - literature covers global calculation, regional (HTAP), and individual megacities
%  - we will show that the effect varies significantly with location of emissions

%Accounting of the effects of NOx, CO, and NMHC emissions on the methane sink, and therefore its radiative forcing, has lagged the discussion of regional effects of these precursors on ozone. Particularly, the spatially-dependent effects of these emissions on total radiative forcing is difficult to calculate efficiently. Previous studies have calculated the sensitivity of methane lifetime to precursor emissions~\citep{ref:prather2001,ref:fry2012,ref:holmes2013}, but lack the ability to calculate these at better than global or regional resolution~\citep{ref:naik2005,ref:macintosh2015}. Full chemistry-climate model studies have been performed for individual megacities~\citep{ref:dang2015}, but the computational cost of perturbing every location individually remains prohibitive. 

%Performing this calculation using a variational method with the adjoint of a chemical transport model provides unprecedented spatial resolution. Furthermore, because the adjoint calculation separates the sensitivity to a change in precursor emissions from the emissions themselves, we can use the sensitivities to efficiently calculate radiative forcing outcomes offline for arbitrary emissions projections. \textbf{KB: this assumes the reader understands the methodology.  Needs to be introduced.}

We conduct 