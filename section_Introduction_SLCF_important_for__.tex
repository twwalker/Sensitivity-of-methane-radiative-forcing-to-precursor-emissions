\section{Introduction}

% SLCF - important for climate
%  - methane is second most important contributor to RF (26% of CO2 forcing)*
%  - ozone is third most important (22% of CO2 forcing)*
%  * according to AR5, here ref:myhre2013

% Precursors of ozone formation
%  - climate, air quality co-benefit
%  - affect ozone formation, also alter local hydroxyl radical concentrations
%  - methane loss in particular is dominated by hydroxyl driven loss in troposphere

Short-lived climate forcers alter the Earth's radiative balance on timescales of a decade or less. Pollutants such as NOx, CO, and non-methane hydrocarbons (NMHC's) affect the Earth's radiative balance not only through ozone formation, but also alter the local hydroxyl radical concentration, which affects the lifetimes of longer-lived greenhouse gases. These precursor gases are also important for air quality, and have significant climate co-benefits~\citep{ref:driscoll2015}. Methane loss in particular is dominated by hydroxyl-driven loss in the troposphere~\citep{ref:kirschke2013}. Methane is the second-greatest anthropogenically produced gas contributor to radiative forcing, with about 26\% the present day forcing of CO2~\citep{ref:myhre2013}.

% OH responds to many factors
%  - primarily methane concentraionts, but these vary slowly
%  - ozone precursors alter local hydroxyl chemistry
%  - changes in photolysis rates through stratospheric ozone burden, clouds, aerosols not covered here

Hydroxyl concentrations in the ACCMIP chemistry climate simulations~\citep{ref:voulgarakis2013} responded primarily to changes in methane concentrations, but also to changes in NOx, NMHC's, and photolysis rates. Small variations in OH help explain interannual variations in methane growth rates~\citep{ref:mcnorton2016}, including the recent hiatus in the growth rate of atmospheric methane concentrations. Importantly, variability in OH concentrations is more significant than changes in temperature in explaining methane growth rates.

% effect of NOx, CO, NMHC on CH4 lifetimes only known at coarse resolution
%  - calculation requires fully resolved tropospheric chemistry simulation (expensive)
%  - literature covers global calculation, regional (HTAP), and individual megacities
%  - we will show that the effect varies significantly with location of emissions

Accounting of precursor emissions has been lacking in policy discussions, as the provincial/city level effects of these emissions on total radiative forcing (which depends on location) is difficult to calculate efficiently. Previous studies have calculated the sensitivity of methane lifetime to precursor emissions~\citep{ref:prather2001,ref:fry2012,ref:holmes2013}, but lack the ability to calculate these at better than global or regional resolution~\citep{ref:naik2005,ref:macintosh2015}. Full chemistry-climate model studies have been performed for individual megacities~\citep{ref:dang2015}, but the computational cost of perturbing every location individually remains prohibitive. Performing this calculation with the adjoint of a chemical transport model provides unprecedented resolution.

% outline of document?
